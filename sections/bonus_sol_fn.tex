%!TEX root = ../main.tex

\setbeamercolor{background canvas}{bg=LightCyan}
\begin{frame}{}
    \begin{center}
        \huge Example: Discontinuous Regularization Paths
    \end{center}
\end{frame}
\setbeamercolor{background canvas}{bg=white}

\begin{frame}{Example: Discontinuous Paths}
    Consider training a toy neural network: given \( (x_1, y_1), (x_2, y_2) \),
    \pause
    \[
        \min_{w_1} \, \half ((w_1 x_1)_+ - y_1)^2 + \half ((w_1 x_2)_+ - y_2)^2 + \lambda |w_1|.
    \]
    \pause

    \begin{center}
        %! TEX root = ../../main.tex

%% Illustration of cone decomposition. 

\begin{tikzpicture}[scale=1,
		declare function={
				loss(\x)= (4*(\x - 4)^2 + 4)*(x<=4.5) + (8*(\x - 5)^2 + 3)*(x>4.5);
			}
	]
	\begin{axis}[width=\linewidth, height=6cm,
			axis lines=center, yticklabels={,,}, xticklabels={,,},
			ymin=-2, ymax=6, ytick={-2,...,5}, ylabel=$$, x axis line style={-},
				xmin=-6, xmax=6, xtick={-5,...,5}, xlabel=$$, y axis line style={-},
		]

		\addplot[name path=loss, domain=-6:6, samples=200, line width=1pt, color=bad]{loss(x)};

		%% point labels
		% origin point
		\node[circle, fill, inner sep=1pt] at (axis cs:0,0) {};

		% active examples 
		\node[label=right:$(x_1\!\mathbin{,} y_1)$, circle, fill, inner sep=0.5mm] at (axis cs:-5,1) {};
		\node[label=right:$(x_2\!\mathbin{,} y_2)$, circle, fill, inner sep=0.5mm] at (axis cs:0.5,5) {};
	\end{axis}

\end{tikzpicture}%

    \end{center}

    \begin{center}
        \Large
        \textcolor{white}{Goal: Overcome these problems via convexification..}
    \end{center}

\end{frame}

\begin{frame}{Example: Discontinuous Paths}
    Consider training a toy ReLU network:
    \[
        \min_{w_1} \, \half ((w_1 x_1)_+ - y_1)^2 + \half ((w_1 x_2)_+ - y_2)^2 + \lambda |w_1|.
    \]

    \begin{center}
        \input{assets/discontinuous_network_0}
    \end{center}

    \begin{center}
        \Large
        \textcolor{white}{Goal: Overcome these problems via convexification..}
    \end{center}

\end{frame}

\begin{frame}{Example: Discontinuous Paths}
    Consider training a toy neural network:
    \[
        \min_{w_1} \, \half ((w_1 x_1)_+ - y_1)^2 + \half ((w_1 x_2)_+ - y_2)^2 + \lambda |w_1|.
    \]

    \begin{center}
        %! TEX root = ../../main.tex

%% Illustration of cone decomposition. 

\begin{tikzpicture}[scale=1,
		declare function={
				loss(\x)= (4*(\x - 4)^2 + 4)*(x<=4.5) + (6*(\x - 5)^2 + 3.5)*(x>4.5);
			}
	]
	\begin{axis}[width=\linewidth, height=6cm,
			axis lines=center, yticklabels={,,}, xticklabels={,,},
			ymin=-2, ymax=6, ytick={-2,...,5}, ylabel=$$, x axis line style={-},
				xmin=-6, xmax=6, xtick={-5,...,5}, xlabel=$$, y axis line style={-},
		]

		\addplot[name path=loss, domain=-6:6, samples=200, line width=1pt, color=bad]{loss(x)};
		\node[label=right:$w_1^*$, circle, fill, inner sep=1pt] at (axis cs:5,3.5) {};

		\draw [name path=cone_1, solid, line width=1pt] (axis cs:0,-2) -- (axis cs:0,6);
		\draw [name path=bounds, line width=0pt] (axis cs:6,-2) -- (axis cs:6,6);
		\tikzfillbetween[of=cone_1 and bounds, on layer=ft]{good, opacity=0.1};

		%% point labels
		% origin point
		\node[circle, fill, inner sep=1pt] at (axis cs:0,0) {};

		% active examples 
		\node[label=right:$(x_1\!\mathbin{,} y_1)$, circle, draw, line width=0.25mm, inner sep=0.5mm] at (axis cs:-5,1) {};
		\node[label=right:$(x_2\!\mathbin{,} y_2)$, circle, fill, inner sep=0.5mm] at (axis cs:0.5,5) {};


		% lines
		\draw [->, draw=bad, line width = 1mm] (axis cs:0,0) -- (axis cs:3,0) node[near end,above] {$w^*_{1}$};

        \node[] at (axis cs:-4,5) { $ \lambda \uparrow $ };
	\end{axis}

\end{tikzpicture}%

    \end{center}

    \begin{center}
        \Large
        \textcolor{white}{Goal: Overcome these problems via convexification..}
    \end{center}

\end{frame}

\begin{frame}{Example: Discontinuous Paths}
    Consider training a toy neural network:
    \[
        \min_{w_1} \, \half ((w_1 x_1)_+ - y_1)^2 + \half ((w_1 x_2)_+ - y_2)^2 + \lambda |w_1|.
    \]

    \begin{center}
        \input{assets/discontinuous_network_2}
    \end{center}

    \pause

    \begin{center}
        \Large
        \good{Goal}: Overcome these problems via convexification.
    \end{center}

\end{frame}

\setbeamercolor{background canvas}{bg=LightCyan}
\begin{frame}{}
    \begin{center}
        \huge Optimal Sets
    \end{center}
\end{frame}
\setbeamercolor{background canvas}{bg=white}

\begin{frame}{Bonus: C-ReLU Optimality Conditions}

    We form the Lagrangian for the convex reformulation:

    \begin{equation*}
        \begin{aligned}
            \calL(v, w, \rho^{+}, \rho^{-})
             & = \half \norm{\sum_{D_i \in \tilde \calD} D_i X (v_i - w_i) - y}_2^2
            + \lambda\sum_{D_i \in \tilde \calD}\norm{v_i}_2 + \norm{w_i}_2                  \\
             & \quad \quad - \sum_{D_i \tilde \calD} \sbr{\abr{\tilde{X_i}^\top \rho^{-}, w}
                - \abr{\tilde{X_i}^\top \rho^{+}, v}},
        \end{aligned}
    \end{equation*}

    where \( \tilde X_{i} = (2D_i - I) \).

    \pause
    \horizontalrule

    The \good{KKT conditions} are necessary and sufficient for optimality:\pause

    \vspace{1ex}
    \begin{itemize}
        \item Stationary Lagrangian:
              \[
                  \underbrace{X^\top D_i (\hat y - y) - \tilde{X_i}^\top \ri^{+}}_{q_i^+} \in \partial \lambda \norm{v_i}_2.
              \]
              \pause
              \begin{itemize}
                  \normalsize
                  \item It turns out each \( q_i^+ \) is \textbf{unique} WLOG!
              \end{itemize}
    \end{itemize}

\end{frame}

\begin{frame}{Bonus: Characterizing the Optimal Set}

    \textbf{Facts}: let \( (\theta, \rho) \) be primal dual optimal. \pause
    \begin{itemize}
        \item Model fit \( \hat y \) is \blue{constant} over optimal set \( \solfn(\lambda) \). \pause
        \item Implies correlation \( X^\top D_i (y - \hat y) \) is \blue{constant} over \( \solfn(\lambda) \). \pause
        \item We may assume \( \rho \) is \blue{unique} (e.g. min-norm dual solution).
    \end{itemize}

    \pause
    \horizontalrule

    \textbf{Non-zero Blocks}:
    \begin{itemize}
        \item Suppose \( \theta_i \neq 0 \).
              \pause
        \item Then \( \nabla \norm{\theta_i}_2 = s_\bi = \lambda \theta_i / \norm{\theta_i}_2 \).
              \pause
        \item Rearranging stationarity implies \( \exists \alpha_\bi > 0 \):
              \[
                  \theta_i = \blue{\alpha_\bi} \underbrace{\sbr{X^\top D_i (y - \hat y) - \tilde X_i \ri}}_{q_i}.
              \]
              \pause
        \item Every solution is a non-negative multiple of these \( q_i \) vectors.
    \end{itemize}

\end{frame}

\begin{frame}{Bonus: Explicit Optimal Set}
    We gave a characterization of \( \solfn(\lambda) \) that depends on
    \[
        \calS_\lambda
        = \cbr{\bi \in [2p] : \exists \theta \in \solfn(\lambda), \
            \theta_i \neq 0}.
    \]

    Alternative expression involves additional linear constraints.

    \pause
    \horizontalrule

    \begin{equation*}
        \begin{aligned}
            \solfn(\lambda) =
            \big\{ & \theta  :
            \forall \, \bi  \in  \equi,
            \theta_i =  \alpha_\bi q_i, \alpha_\bi \geq 0, \,           \\
                   & \quad \forall \, j \in [2p] \setminus \equi,
            \theta_{j} = 0, \, \sum_{i=1}^{2p} D_i X \theta_i = \hat y, \\
                   & \quad \forall \, i \in [2p],
            \tilde X_i \theta_i \geq 0, \abr{\rho, \tilde X_i \theta_i } = 0.
            \big\}
        \end{aligned}
    \end{equation*}

    \pause

    More complex, but also \textbf{explicit}.

\end{frame}

\begin{frame}{Bonus: Solution Mapping for C-ReLU}

    Given \( \rbr{v^*, w^*} \), an optimal non-convex ReLU network is given by

    \begin{equation*}
        \textbf{C to NC:} \quad \quad
        \begin{aligned}
            W_{1i} & = v_i^*/ \sqrt{\norm{v_i^*}}, \quad w_{2i} = \sqrt{\norm{v_i^*}}
            \\
            W_{1j} & = w_i^*/ \sqrt{\norm{w_i^*}}, \quad w_{2j} = -\sqrt{\norm{w_i^*}}.
        \end{aligned}
    \end{equation*}

    \pause
    \vspace{3ex}
    \begin{itemize}
        \item Optimal convex weights satisfy \( v_i^* = \alpha_i q_i \)
              so that
              \[
                  \norm{v_i^*}_2 = \alpha_i \norm{q_i}_2 = \alpha_i \lambda.
              \]
    \end{itemize}

    \pause
    \horizontalrule

    Recall structure of \textbf{non-convex optima}:

    \begin{equation*}
        \begin{aligned}
            \hspace{-0.5em} \calO_\lambda  = \,
            \big\{
             & (W_1,  w_2) :
            \, f_{W_1, w_2}(X)  =  \hat y,                       \\
             & \forall \, \bi  \in  \calS_\lambda,
            W_{1i} = (\sfrac{\alpha_{i}}{\lambda})^{\sfrac{1}{2}} q_i,
            w_{2i} = (\alpha_i \lambda)^{\sfrac{1}{2}},
            \alpha_i \geq 0                                      \\
             & \forall \, \bi  \in [2p] \setminus \calS_\lambda,
            W_{1i} = 0, \, w_{2i} = 0
            \big\}.
        \end{aligned}
    \end{equation*}

\end{frame}

\setbeamercolor{background canvas}{bg=LightCyan}
\begin{frame}{}
    \begin{center}
        \huge Optimal Pruning
    \end{center}
\end{frame}
\setbeamercolor{background canvas}{bg=white}

\begin{frame}{Optimal Pruning: the Polytope of Solutions}

    \begin{columns}
        \begin{column}{0.5\textwidth}

            \begin{equation*}
                \begin{aligned}
                     & \solfn(\lambda) =
                    \big\{ \theta  : \sum_{i=1}^{2p} D_i X \theta_i = \hat y,      \\
                     & \hspace{5em} \forall \, \bi  \in  \calS_\lambda,
                    \theta_i =  \alpha_\bi q_i, \alpha_\bi \geq 0, \,              \\
                     & \hspace{5em} \forall \, j \in [2p] \setminus \calS_\lambda,
                    \theta_{j} = 0
                    \big\}
                \end{aligned}
            \end{equation*}

            \pause
            The C-ReLU optimal set is a \good{convex polytope}!

        \end{column}
        \begin{column}{0.5\textwidth}
            \pause
            \begin{figure}[c]
                \centering
                \includegraphics[width=0.8\textwidth]{assets/polytope.png}
            \end{figure}
        \end{column}
    \end{columns}

\end{frame}

\begin{frame}{Optimal Pruning: Vertices}

    \begin{enumerate}
        \item
              Stack the \( q_i \) vectors into a matrix \( Q =
              \begin{bmatrix}
                  \vert &        & \vert  \\
                  q_1   & \cdots & q_{2p} \\
                  \vert &        & \vert  \\
              \end{bmatrix}.
              \)

              \pause
        \item
              The C-ReLU Optimal Set in \( \alpha \) space is then,
              \begin{equation}
                  \begin{aligned}
                      \solfn(\lambda) & =
                      Q_{\calS_\lambda} \big\{ \alpha \succeq 0  :
                      \sum_{i \in \calS_\lambda} (D_i X q_i) \alpha_i = \hat y,
                      \big\}                                                       \\
                                      & = Q_{\calS_\lambda} \calP_{\calS_\lambda}.
                  \end{aligned}
              \end{equation}

              \pause

        \item \( \bar \alpha \in \calP_{\calS_\lambda} \) is a \good{vertex}
              iff \( \cbr{D_i X q_i}_{\bar \alpha_i \neq 0} \) are linearly independent.
    \end{enumerate}

    \pause
    \horizontalrule

    {\large Are these vertices \good{special} in some way?}

\end{frame}

\begin{frame}{Optimal Pruning: Minimal Models}
    \textbf{Definition}: An optimal C-ReLU model \( \theta^* \) is minimal if
    there does not exist another optimal model \( \theta' \) with \bad{strictly
        smaller support}.

    \pause
    \begin{itemize}
        \item \bad{NC-ReLU}: minimal \( \iff \) \good{sparsest} (neuron-wise) model.
    \end{itemize}

    \vspace{3ex}
    \pause

    \begin{beamercolorbox}[wd=\textwidth,sep=1em]{result}
        \textbf{Proposition 3.2} (informal):
        For \( \lambda > 0 \), \( \theta \in \solfn(\lambda) \) is \good{minimal}
        iff
        the vectors \( \cbr{D_i X q_i}_{\alpha_i \neq 0} \)
        are linearly independent.
    \end{beamercolorbox}

    \pause

    \begin{itemize}
        \item \bad{NC-ReLU}: minimal if \( (X W_{1i})_+ \) are linearly independent.
    \end{itemize}

    \vspace{1ex}
    \pause

    \textbf{Our Results}:
    \begin{enumerate}
        \item We prove vertices of \( \solfn(\lambda) \) are minimal models.
              \pause
        \item There are at most \( n \) neurons in a minimal model.
              \pause
        \item We give a poly-time algorithm for computing minimal models
              starting from any model \( \theta \).
    \end{enumerate}

\end{frame}

\begin{frame}{Bonus: Optimal Pruning Pseudo-code}
    \begin{algorithm}[H]
        \caption{Pruning solutions}
        \begin{algorithmic}
            \STATE {\bfseries Input:} data matrix \( X \), solution \( \theta \).
            \STATE \( k \gets 0 \).
            \STATE \( \theta^k \gets \theta \).
            \WHILE {\( \exists \beta \neq 0 \) s.t. \( \sum_{\bi \in \act(\theta^k)} \beta_\bi D_i X \theta_i^k = 0 \)}
            \STATE \( \bi^k \gets \argmax_{\bi} \cbr{|\beta_\bi| : \bi \in \act(\theta^k)}  \)
            \STATE \( t^k \gets 1/|\beta_{\bi^k}| \)
            \STATE \( \theta^{k+1} \gets \theta^k (1 - t^k \beta_\bi) \)
            \STATE \( k \gets k + 1 \)
            \ENDWHILE
            \STATE {\bfseries Output:} final weights \( \theta^k \)
        \end{algorithmic}
    \end{algorithm}

    \pause

    Let \( r = \text{rank}(X) \). Complexity to compute a minimal model:

    \[ O\rbr{d^3 r^3 (\frac{n}{r})^{3r} + \blue{(n^3 + nd) r (\frac{n}{r})^r}}. \]

\end{frame}

\begin{frame}{Bonus: Complexity of Pruning}

    \begin{algorithm}[H]
        \caption{Pruning solutions}
        \begin{algorithmic}
            \STATE {\bfseries Input:} data matrix \( X \), solution \( \theta \).
            \STATE \( k \gets 0 \), \( \theta^k \gets \theta \).
            \WHILE {\( \exists \beta \neq 0 \) s.t. \( \sum_{\bi \in \act(\theta^k)} \beta_\bi D_i X \theta_i^k = 0 \)}
            \STATE \( \vdots \)
            \ENDWHILE
            \STATE {\bfseries Output:} final weights \( \theta^k \)
        \end{algorithmic}
    \end{algorithm}

    \begin{itemize}
        \item Computing \( a_i = D_i X \theta_i^0 \) for every neuron: \( O(ndp)  \)
              \pause
              \vspace{1ex}
        \item Checking for linear dependence: at most \( 2p \) times, do
              \pause
              \vspace{1ex}
              \begin{itemize}
                  \item check (at most) \( n+1 \) \( a_i \) vectors for linearly dependence.
                        \pause
                  \item Form matrix \( A \) and take SVD to compute null space: \( O(n^3) \).
                        \pause
                  \item Prune neuron: update at most \( n \) weights.
                        \pause
                        \vspace{1ex}
              \end{itemize}
    \end{itemize}
    Total complexity: \( O(ndp + n^3 p) \).

\end{frame}

\begin{frame}{Bonus: Sub-Optimal Pruning}

    \begin{algorithm}[H]
        \caption{Pruning solutions}
        \begin{algorithmic}
            \STATE {\bfseries Input:} data matrix \( X \), solution \( \theta \).
            \STATE \( k \gets 0 \).
            \STATE \( \theta^k \gets \theta \).
            \WHILE {\( \exists \beta \neq 0 \) s.t. \( \bad{\sum_{\bi \in \act(\theta^k)} \beta_\bi D_i X \theta_i^k = 0} \)}
            \STATE \( \bi^k \gets \argmax_{\bi} \cbr{|\beta_\bi| : \bi \in \act(\theta^k)}  \)
            \STATE \( t^k \gets 1/|\beta_{\bi^k}| \)
            \STATE \( \theta^{k+1} \gets \theta^k (1 - t^k \beta_\bi) \)
            \STATE \( k \gets k + 1 \)
            \ENDWHILE
            \STATE {\bfseries Output:} final weights \( \theta^k \)
        \end{algorithmic}
    \end{algorithm}

    \pause
    Approximate with least-squares fit:
    \[
        \hat{\beta} \in \argmin_{\beta} \half \norm{\sum_{\bi \in \act(\theta^k) \setminus j} \beta_\bi D_i X \theta_i^k - D_j X \theta_j}_2^2
    \]
\end{frame}

\begin{frame}{Bonus: Sub-optimal Pruning}
    Approximate with least-squares fit:
    \[
        \hat \beta \in \argmin_{\beta} \half \norm{\sum_{\bi \in \act(\theta^k) \setminus j} \beta_\bi D_i X \theta_i^k - D_j X \theta_j}_2^2
    \]

    \horizontalrule

    \begin{itemize}
        \item Algorithm is just structured pruning with a \good{correction step}!
              \pause
        \item We use \good{existing literature} for structured pruning to select \( j \).
              \pause
        \item \bad{Brute-force search} works best:
              \[
                  \argmin_{j} \cbr{\min_{\beta} \half \norm{\sum_{\bi \in \act(\theta^k) \setminus j} \beta_\bi D_i X \theta_i^k - D_j X \theta_j}_2^2}
              \]
    \end{itemize}

\end{frame}

\begin{frame}{Neuron Pruning: Performance on CIFAR-10}
    \begin{figure}[t]
        \centering
        \includegraphics[width=0.75\textwidth]{assets/prune_cifar.pdf}
    \end{figure}
\end{frame}

\setbeamercolor{background canvas}{bg=LightCyan}
\begin{frame}{}
    \begin{center}
        \huge Convex Reformulations of Deep ReLU Networks
    \end{center}
\end{frame}
\setbeamercolor{background canvas}{bg=white}

\begin{frame}{Deep Reformulations: Setup}

    \begin{itemize}
        \item Let \( X^{(l)} \) and \( T^{(l)} \) be tensors of order
              \( l+1 \) indexed by \( i_0, \ldots, i_{l} \).

              \vspace{1ex}

        \item We assume  \( R^{(l)}_{i_1, \ldots, i_{l-1}} \in \R^{n \times d_0} \) and \( T^{(l)}_{i_1, \ldots, i_{l-1}} \in \R^{d_0 \times d_{l}} \).

    \end{itemize}

    \horizontalrule

    We equip these tensors with the reduction product
    \begin{equation*}\label{eq:reduction-prod}
        R^{(l)} \odot T^{(l)}
        = \sum_{i_1, \ldots, i_{l-1}} R^{(l)}_{i_1, \ldots, i_{l-1}}
        T^{(l)}_{i_1, \ldots, i_{l-1}},
    \end{equation*}

    \begin{itemize}
        \item This sums over the product of all the matrix slices
              \( R^{(l)}_{i_1, \ldots, i_{l-1}} \) and
              \( T^{(l)}_{i_1, \ldots, i_{l-1}} \).
    \end{itemize}

\end{frame}

\begin{frame}{Deep Reformulations: Recursive Patterns}
    \begin{itemize}
        \item Let \( \calD^{(1)}_X \) be the set of achievable
              ReLU patterns in the first layer.

        \item Let \( X^{(1)} = X \in \R^{n \times d_0} \).

        \item Define
              \( X^{(l+1)}_{i_1, \ldots, i_{l}}
              = D^{(l)}_{i_{l}} X^{(l)}_{i_1, \ldots, i_{l-1}} \)
              so that we have,
              \[
                  \begin{aligned}
                      X^{(2)}_{i_1}
                       & = D^{(1)}_{i_1} X^{(1)} = D^{(1)}_{i_1} X                      \\
                      X^{(3)}_{i_1, i_2}
                       & = D^{(2)}_{i_2} X^{(2)}_{i_1} = D^{(2)}_{i_2} D^{(1)}_{i_1} X.
                  \end{aligned}
              \]

        \item Here, \( D_{i_l}^{(l)} \in \calD^{(l)}_X \) is the set of
              ReLU patterns achievable by our tensor product in the
              \( l^{\text{th}} \) layer,
              \[
                  \calD_{X}^{(l)}
                  = \cbr{ \mathds{1}\rbr{ X^{(l)} \odot T^{(l)} \geq 0 } : T^{(l)} \in \R^{d_0 \times \cdots \times d_{l}} }.
              \]
    \end{itemize}

\end{frame}

\begin{frame}{Deep Reformulations: Tensor Decomposition}

    Each tensor \( T^{(l)}_{i_{l}} \) is contained in at least
    one activation cone,
    \begin{equation*}\label{eq:tensor-cone}
        \calK^{(l)}_{i_{l}} =
        \cbr{
        T^{(l)}_{i_l} \in \R^{d_0 \times \cdots \times d_{(l-1)}}
        :
        (2D^{(l)}_{i_{l}} - I)
        \sbr{X^{(l)} \odot T^{(l)}_{i_{l}}} \geq 0
        }.
    \end{equation*}

    Now, let \( \calG^{(1)} = \R^{d_0 \times d_1} \) and define
    \begin{equation*}\label{eq:rank-tensor-decomp}
        \begin{aligned}
            \calG^{(l+1)}
            :=
              & \bigg\{
            T^{(l+1)} \in \R^{d_{0} \times p_{1} \ldots \times p_{l} \times d_{l+1}}  \\
            : & \exists \, T^{(l)} \in \calG^{(l)} \text{ where }
            \calI^{(l)}_{j_{l}} = \cbr{ i_l : T^{(l)}_{i_l} \in \calK^{(l)}_{j_{l}}}, \\
              & \exists \, W^{(l+1)} \in \R^{d_l \times d_{l+1}},                     \\
              & \text{s.t. } T^{(l+1)}_{j_1, \ldots, j_{l}}
            = \sum_{i_l \in \calI^{(l)}_{j_l}} T^{(l)}_{j_1, \ldots, j_{l-1}, i_{l}}
            \otimes
            W^{(l+1)}_{i_l},                                                          \\
              & \text{and } \sum_{j_l = 1}^{p_l} \abs{\calI^{(l)}_{j_l}} \leq d_{l}
            \bigg\}.
        \end{aligned}
    \end{equation*}
\end{frame}

\begin{frame}{Deep Reformulations: Layer-Merging Lemma}

    \begin{lemma}[Rank-Controlled Layer Elimination]
        Let \( T^{(l)} \in \calG^{(l)} \).
        Then the activations at layer \( l+2 \) are given by
        \begin{equation}\label{eq:layer-to-tensorize}
            Z^{(l+2)} =
            \rbr{\sum_{i_l=1}^{d_l} \rbr{X^{(l)} \odot T^{(l)}_{i_l}}_+
                W^{(l+1)}_{i_l}}_+,
        \end{equation}
        if and only if the activations are also equal to
        \[
            Z^{(l+2)} = \rbr{X^{(l+1)} \odot T^{(l+1)}}_+,
        \]
        for some \( T^{(l+1)} \in \calG^{(l+1)} \).
    \end{lemma}
\end{frame}


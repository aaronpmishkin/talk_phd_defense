%!TEX root = ../main.tex

\setbeamercolor{background canvas}{bg=LightCyan}
\begin{frame}{}
    \begin{center}
        \huge Additional Projects
    \end{center}
\end{frame}
\setbeamercolor{background canvas}{bg=white}

\begin{frame}{Directional Smoothness: Summary}
    \citep{mishkin2024directional}
    Directional Smoothness and Gradient Methods: Convergence and Adaptivity.
    \good{A. Mishkin}*, A. Khaled*, Y. Wang, A. Defazio, R. M.  Gower. NeurIPS
    2024.

    \horizontalrule

    \pause
    \textbf{Big Idea}: GD is a local algorithm, so the analysis should be
    local.
    \vspace{1ex}

    \pause
    \textbf{Main Contributions}:
    \pause
    \vspace{0.5ex}
    \begin{itemize}
        \item \good{Directional Smoothness}: concrete functions \( M(x, y) \)
              that measure the Lipschitz smoothness of \( f \) between \( x \)
              and \( y \).
              \pause
              \vspace{1ex}

        \item \good{Path-Dependent Rates}: Convergence bounds for gradient descent
              that depend only on \( M(\xkk, \xk) \).
              \pause
              \vspace{1ex}

        \item \good{Adaptive Methods}: algorithms which are ``strongly adaptive''
              to the directional smoothness along the optimization path.
    \end{itemize}
\end{frame}

\begin{frame}{Level Set Teleportation: Summary}
    \citep{mishkin2025level}
    Level Set Teleportation: An Optimization Perspective. \good{A.  Mishkin},
    A. Bietti, R. M. Gower. AISTATS 2025.

    \horizontalrule
    \pause

    \textbf{Big Idea}: rigorously analyze and experimentally evaluate
    Newton-like ``teleportation'' methods \citep{zhao2022symmetry}.

    \vspace{1ex}

    \pause
    \textbf{Main Contributions}:
    \pause
    \vspace{0.5ex}
    \begin{itemize}
        \item \good{Sub-Level Set Teleportation}:
              a co-routine to accelerate optimization by maximizing the gradient
              over a sub-level set,
              \[
                  x_k^+ = \argmin_x \cbr{\norm{\grad(x)}_2 : f(x) \leq f(x_k)}.
              \]
              \pause

        \item We give a novel analysis combining \good{linear progress}
              from teleportation with sub-linear
              rates for non-strongly convex GD.

              \pause
              \vspace{1ex}

        \item We also develop \good{parameter-free algorithms} for solving
              general, non-linear sub-level set teleportation problems in
              practice.
    \end{itemize}
\end{frame}

\begin{frame}{Glocal Smoothness: Summary}
    \citep{fox2025glocal}
    Glocal Smoothness: Line Search can really help! C. Fox, \good{A.  Mishkin},
    S. Vaswani, M. Schmidt. SIOPT (\red{In Review}).

    \horizontalrule

    \pause
    \textbf{Big Question}: Can we show provable speed-ups from line-search?
    \vspace{1ex}

    \pause
    \textbf{Main Contributions}:
    \pause
    \vspace{0.5ex}
    \begin{itemize}
        \item \good{Glocal smoothness}: functions which are globally
              \( L \)-Lipschitz smooth and locally \( L^* \)-smooth
              (\( L^* \ll L \)) around minimizers.
              \pause
              \vspace{1ex}

        \item Convergence rates showing GD with exact line-search and GD with
              the Polyak step-size \good{adapt} to glocal smoothness.

              \pause
              \vspace{1ex}

        \item Extensions to \good{Nesterov acceleration}, with and without
              a backtracking line-search.
    \end{itemize}
\end{frame}

\begin{frame}{Greedy Block-Coordinate Descent: Summary}
    \citep{ramesh2023greedy}
    Greedy 2-Coordinate Updates for Equality-Constrained Optimization via Steepest
    Descent in the 1-Norm. A. V. Ramesh*, \good{A. Mishkin}*, M. Schmidt, et al.
    (\red{To Be Submitted})

    \horizontalrule
    \pause

    \textbf{Problem}: greedy rules out-perform random selection for training
    SVMs, but rates don't reflect this.

    \vspace{1ex}

    \pause
    \textbf{Main Contributions}:
    \pause
    \vspace{0.5ex}
    \begin{itemize}
        \item New analyses of \good{block-coordinate descent} for separable
              optimization problems with linear coupling constraints.
              \pause
              \vspace{1ex}

        \item \good{Single Constraint}: We prove that the GS-q rule
              \citep{tseng2009block} is equivalent to steepest descent in the
              \( \ell_1 \)-norm, leading to fast rates with \( \mu_1 \)
              dependence.

              \pause
              \vspace{1ex}

        \item \good{Multiple Constraints}: We extend the conformal vector
              framework used by \citet{tseng2009block} to obtain rates
              which improve with block-size.
    \end{itemize}
\end{frame}

\begin{frame}{Gradient Flows: Summary}
    Global Convergence of Gradient Flow on 1D Data with Sign Noise.  \good{A.
        Mishkin}, F. Bach. (\red{Ongoing})

    \horizontalrule
    \pause

    \textbf{Big Idea}: understand how ``neuron diversity'' affects convergence
    of the gradient flow for shallow ReLU networks.

    \vspace{1ex}

    \pause
    \textbf{Main Contributions}:
    \pause
    \vspace{0.5ex}
    \begin{itemize}
        \item \good{Diversity}: model neurons have different activation patterns.
              \pause
              \vspace{1ex}

        \item We find diversity is preserved for simple anti-correlated noise
              and the gradient flow is \good{globally convergent}.
              \pause
              \vspace{1ex}

        \item The gradient flow demonstrates \good{saddle-to-saddle dynamics}
              \citep{pesme2023saddle}, with saddles corresponding to model
              complexity.
              \pause
              \vspace{1ex}
    \end{itemize}

    \textbf{Remaining Extensions}:
    \pause
    \begin{enumerate}
        \item Generalize the target noise to more \bad{interesting/realistic}
              regimes (e.g. fractional Brownian motion).
    \end{enumerate}
\end{frame}

\setbeamercolor{background canvas}{bg=LightCyan}
\begin{frame}{}
    \begin{center}
        \huge Background on Convex Neural Networks
    \end{center}
\end{frame}
\setbeamercolor{background canvas}{bg=white}

\begin{frame}{Bonus: Convex Neural Networks}
    \begin{itemize}
        \item Let \( X \in \R^{n \times d} \) be a training matrix and \( y \in \R^n \)
              the targets.
              \vspace{1ex}
              \pause

        \item Let \( f_\theta : \R^n \into \R \) be a neural network with parameters
              \( \theta \).
    \end{itemize}
    \vspace{1ex}
    \pause
    The standard ERM training problem is,
    \[
        \Theta^* = \argmin_{\theta} \sum_{i=1}^n \mathcal{L}(f_\theta(x_i), y_i) + r(\theta),
    \]
    where \( \calL \) is a loss function and \( r \) is a regularizer.
    \pause
    \vspace{1ex}

    \begin{itemize}
        \item \( f_\theta \) is a \good{convex neural network} if this problem
              is convex in \( \theta \).
              \vspace{1ex}
              \pause
        \item This is distinct from \bad{input-convex} neural networks, where
              \( x \mapsto f_\theta(x) \) is convex \citep{amos2017input}.
    \end{itemize}

\end{frame}

\begin{frame}{Bonus: Brief Literature Review}
    There are multiple flavours of convex neural networks:
    \vspace{1ex}

    \begin{itemize}
        \item \citet{bengio2005convex} develop a \good{gradient-boosting algorithm}
              where the problem of adding one neuron at a time is convex.
              \pause
              \vspace{1ex}

        \item \citet{bach2017breaking} study the generalization performance of
              \good{infinite-width} two-layer neural networks (which are convex).
              \pause
              \vspace{1ex}

        \item \citet{pilanci2020convex} develop finite-width \textbf{convex
                  reformulations} for two-layer ReLU networks using \good{duality}.
              \pause
    \end{itemize}
    \vspace{1ex}

    These approaches differ primarily in how they discretize the underlying
    infinite-width neural network.

\end{frame}

\begin{frame}{Bonus: Function Space Viewpoint}
    \citet{bengio2005convex, bach2017breaking} take a function space approach:

    \begin{itemize}
        \item Let \( \sigma \) be an activation function and define
              \[
                  \calH = \cbr{h : w \in \R^d, h(x) = \sigma(x^\top w)}.
              \]
              \pause

        \item Write problem as optimization over function space \( W \):
              \[
                  \min_{w \in \calW} \cbr{\sum_{j=1}^n L\rbr{\sum_{h_i \in \calH} w_i h_i(x_j), y_j} + R(w)}.
              \]

              \pause

        \item If \( R \) is sparsity inducing, then the final network may have finite width.
    \end{itemize}

\end{frame}

\begin{frame}{Bonus: Related Work Cont.}
    \citet{bengio2005convex}: algorithm-focused approach.
    \vspace{1ex}
    \begin{itemize}
        \item Take \( R(w) = \norm{w}_1 \) and \( L(\hat y, y) = \max\cbr{0, 1 - \hat y y} \).
              \pause
              \vspace{1ex}
        \item Show that \( \text{nnz}(w^*) \leq n+1 \), meaning the final model is finite.
              \pause
              \vspace{1ex}
        \item Propose a boosting-type algorithm for iteratively adding neurons.
    \end{itemize}

\end{frame}
\begin{frame}{Bonus: Related Work Cont.}
    \citet{bach2017breaking}: analysis-focused approach.

    \begin{itemize}
        \item Handle spaces/functions properly using measure theory.
              \vspace{1ex}
              \pause
              \begin{itemize}
                  \item \( \calW \) is a space of signed measures, prediction is
                        \[
                            f(x) = \int_{\calH} h(x) dw(h)
                        \]
                        \vspace{1ex}
                        \pause

                  \item \( R \) is weighted total variation of measure \( w \).
                        \pause
                        \vspace{1ex}

                  \item Setup reduces to \citet{bengio2005convex} in finite spaces.
              \end{itemize}

              \pause
              \vspace{1ex}
        \item Guarantee that \( m^* \leq n \) using a representer theorem.
              \pause
              \vspace{1ex}

        \item Derive an incremental algorithm based on Frank-Wolfe, but incremental steps are NP-Hard
              for ReLU activations.
    \end{itemize}

\end{frame}

\begin{frame}{Bonus: Key Representer Theorem}
    \begin{theorem}[\citet{rogosinski1958moments}]
        If \( \rbr{\Omega, \calB} \) is a Borel space, \( \mu \) is a measure,
        \( g_i \), \( i \in \cbr{1, \ldots n} \) are measurable and \( \mu \)-integrable,
        then there exists measure \( \hat \mu \) with finite support at most \( n \)
        such that
        \[
            \int_\Omega g_i(\omega) d\mu(\omega) = \int_\Omega g_i(\omega) d \hat \mu(\omega)
        \]
        for all \( i \in \cbr{1,\ldots,n} \).
    \end{theorem}

    Prediction for dataset with \( n \) dimensions:
    \[
        f(x_i) = \int_{\calH} h(x_i) dw(h) = \sum_{h = 1}^m h_j(x_i) w(h_j),
    \]
    where \( m \leq n \) and \( h_j(x) = \rbr{\abr{x, w_j}}_+ \).
\end{frame}

\setbeamercolor{background canvas}{bg=LightCyan}
\begin{frame}{}
    \begin{center}
        \huge Convex Reformulations
    \end{center}
\end{frame}
\setbeamercolor{background canvas}{bg=white}

\begin{frame}{Convex Reformulations: Breaking it Down}

    \[
        \begin{aligned}
            \min_{u} & \norm{\sum_{j=1}^p D_j X (v_j - w_j) - y}_2^2 +
            \lambda \sum_{j=1}^p \norm{v_j}_2 + \norm{w_j}_2           \\
                     & \hspace{0.2em} \text{s.t. }
            v_j, w_j \in \calK_j := \cbr{w : (2D_j - I) X w \geq 0},
        \end{aligned}
    \]
    where \( \purple{D_j = \text{diag}[\mathbbm{1}(X g_j \geq 0)]} \).

    \horizontalrule

    \begin{itemize}
        \item \purple{\( D_j \) is a ReLU activation pattern induced by ``gate'' \( g_j \).}
              \pause
              \begin{itemize}
                  \item \([D_j]_{ii} = 1\) if \( \abr{x_i, g_i} \geq 0 \) and \( 0 \) otherwise.
              \end{itemize}

    \end{itemize}
    \vspace{7.3em}
\end{frame}

\begin{frame}{Convex Reformulations: Breaking it Down}

    \[
        \begin{aligned}
            \min_{u} & \norm{\sum_{j=1}^p D_j X (v_j - w_j) - y}_2^2 +
            \purple{\lambda \sum_{j=1}^p \norm{v_j}_2 + \norm{w_j}_2}  \\
                     & \hspace{0.2em} \text{s.t. }
            v_j, w_j \in \calK_j := \cbr{w : (2D_j - I) X w \geq 0}
        \end{aligned}
    \]
    where \( D_j = \text{diag}[\mathbbm{1}(X g_j \geq 0)] \).

    \horizontalrule

    \begin{itemize}
        \item \( D_j \) is a ReLU activation pattern induced by ``gate'' \( g_j \).
              \begin{itemize}
                  \item \([D_j]_{ii} = 1\) if \( \abr{x_i, g_i} \geq 0 \) and \( 0 \) otherwise.
              \end{itemize}
        \item \purple{Weight-decay regularization turns into ``group \( \ell_1 \)'' penalty.}
    \end{itemize}
    \vspace{6em}
\end{frame}

\begin{frame}{Convex Reformulations: Breaking it Down}

    \[
        \begin{aligned}
            \min_{u} & \norm{\sum_{j=1}^p D_j X (v_j - w_j) - y}_2^2 +
            \lambda \sum_{j=1}^p \norm{v_j}_2 + \norm{w_j}_2           \\
                     & \hspace{0.2em} \purple{\text{s.t. }
                v_j, w_j \in \calK_j := \cbr{w : (2D_j - I) X w \geq 0},}
        \end{aligned}
    \]
    where \( D_j = \text{diag}[\mathbbm{1}(X g_j \geq 0)] \).

    \horizontalrule

    \begin{itemize}
        \item \( D_j \) is a ReLU activation pattern induced by ``gate'' \( g_j \).
              \begin{itemize}
                  \item \([D_j]_{ii} = 1\) if \( \abr{x_i, g_i} \geq 0 \) and \( 0 \) otherwise.
              \end{itemize}
        \item Weight-decay regularization turns into ``group \( \ell_1 \)'' penalty.
        \item \purple{The constraint \( v_j \in \calK_j \) implies
                  \[
                      \rbr{X v_j}_+ = D_j X v_j.
                  \]
                  That is, \( v_j \) has the activation encoded by \( D_j \).
              }
    \end{itemize}
\end{frame}

\begin{frame}{Bonus: Explicit Solution Mapping}

    Given \( \rbr{v^*, w^*} \), an optimal non-convex ReLU network is given by

    \begin{equation*}
        \textbf{C to NC:} \quad \quad
        \begin{aligned}
            W_{1i} & = v_i^*/ \sqrt{\norm{v_i^*}}, \quad w_{2i} = \sqrt{\norm{v_i^*}}
            \\
            W_{1j} & = w_i^*/ \sqrt{\norm{w_i^*}}, \quad w_{2j} = -\sqrt{\norm{w_i^*}}.
        \end{aligned}
    \end{equation*}

    \begin{itemize}
        \item Optimal solution balances weight between layers.
    \end{itemize}

    \horizontalrule

    Given \( (W^*_{1i}, w^*_{2i}) \), an optimal convex ReLU model is

    \begin{equation*}
        \textbf{NC to C:} \quad \quad
        \begin{aligned}
            v_i & = W^*_{1i} \abs{w_{2i}}^* \quad \text{ if } w^*_{2i} \geq 0 \\
            w_i & = W^*_{1i} \abs{w_{2i}}^* \quad \text{ Otherwise.}
        \end{aligned}
    \end{equation*}

    \begin{itemize}
        \item Optimal solution combines weight from both layers.
    \end{itemize}

\end{frame}

\setbeamercolor{background canvas}{bg=LightCyan}
\begin{frame}{}
    \begin{center}
        \huge Gated ReLU Networks and Cone Decompositions
    \end{center}
\end{frame}
\setbeamercolor{background canvas}{bg=white}

\begin{frame}{Bonus: Gated ReLU Networks}

    \begin{beamercolorbox}[wd=\textwidth,sep=1em]{result}
        \textbf{Theorem 2.2} (informal): C-GReLU is equivalent to solving
        \[
            \textbf{NC-GReLU}: \min_{W_1, \alpha} \half \norm{\sum_{j=1}^p \phi_{g_j}(X, w_{j})\alpha - y}_2^2 + \frac{\lambda}{2} \sum_{j=1}^p \norm{w_{j}}_2^2 + |\alpha_j|^2,
        \]
        with the ``Gated ReLU'' \citep{fiat2019decoupling} activation function
        \[ \phi_{g}(X, u) = \text{diag}(\mathbbm{1}(Xg \geq 0)) X u, \]
        and gate vectors \( g_j \) such that
        \[
            D_j = \text{diag}[\mathbbm{1}(X g_j \geq 0).
        \]
    \end{beamercolorbox}
    \pause

    \textbf{Interpretation}: if \( u_j \not \in \calK_j \), then the activation
    must be decoupled from the linear mapping in the non-convex model.

\end{frame}

\begin{frame}{Bonus: Cone Decompositions}

    \begin{center}
        \textbf{Question}: when are Gated ReLU and ReLU networks equivalent?
    \end{center}

    \pause
    \horizontalrule

    Consider special case where \( \lambda = 0 \).

    \[
        \textbf{C-GReLU}: \min_{u} \norm{\sum_{j=1}^p D_j X u_j - y}_2^2. \hspace{11em}
    \]
    \vspace{-1em}

    \begin{center}
        \Large \red{V.S.}
    \end{center}

    \vspace{-2em}
    \[
        \begin{aligned}
            \textbf{C-ReLU}: \min_{u} & \norm{\sum_{j=1}^p D_j X (v_j - w_j) - y}_2^2. \\
                                      & \hspace{0.2em} \text{s.t. }
            v_j, w_j \in \calK_j := \cbr{w : (2D_j - I) X w \geq 0},
        \end{aligned}
    \]

\end{frame}

\begin{frame}{Bonus: Equivalent Statement}

    \red{Equiv. Question}: when does \( u_j = v_j - w_j \) for some \( v_j, w_j \in \calK_j \)?

    \pause
    \vspace{1em}

    \green{Answer}: when \( \calK_j - \calK_j = \R^d \) and a ``cone decomposition'' exists.
    \pause

    \begin{figure}[]
        \centering
        %! TEX root = ../../main.tex

%% Illustration of cone decomposition. 

\begin{tikzpicture}[scale=1,
		declare function={
				cone_u(\x)= -\x/3;
				cone_l(\x)= -4*\x/5;
			}
	]
	\begin{axis}[width=1.1\linewidth, height=6cm,
			axis lines=center, yticklabels={,,}, xticklabels={,,},
			ymin=-4, ymax=4, ytick={-5,...,5}, ylabel=$$, x axis line style={-},
				xmin=-6, xmax=6, xtick={-5,...,5}, xlabel=$$, y axis line style={-},
		]
		\addplot[name path=cone_u, domain=-6:6, samples=100, line width=1pt]{cone_u(x)};
		\addplot[name path=cone_l, domain=-6:6, samples=200, line width=1pt]{cone_l(x)};

		% add color fill to both cones.
		\addplot fill between[
				of = cone_u and cone_l,
				split, % calculate segments
				every even segment/.style = {fill=blue, fill opacity=0.3},
				every odd segment/.style  = {fill=teal, fill opacity=0.3}
			];

		%% point labels
		% origin point
		\node[circle, fill, inner sep=1pt] at (axis cs:0,0) {};

		\node[label={0:$u_j$}, circle, fill, inner sep=1.8pt] (u) at (axis cs:2,1) {};
		\node[label={90:$v_j$}, circle, fill, inner sep=1.8pt] (v) at (axis cs:25/7+2, -25/21 - 2/3) {};
		\node[label={90:$w_j$}, circle, fill, inner sep=1.8pt] (w) at (axis cs:-25/7, 20/7) {};

		% labels
		\node[label={0:$\calK_j$}] at (axis cs:4,-2.5) {};
		\node[label={180:$-\calK_j$}] at (axis cs:-3.75,2.5) {};

		% lines
		\draw [->, dashed, draw=red, line width = 0.4mm] (u) edge (w);
		\draw [->, dashed, draw=red, line width = 0.4mm] (u) edge (v);
	\end{axis}

\end{tikzpicture}%

    \end{figure}

\end{frame}

\begin{frame}{Bonus: Basic Cone Decomposition}

    \textbf{Recall}: \( \calK_j = \cbr{w : (2 D_j - I) X w \geq 0} \).
    \pause

    \begin{itemize}
        \item This is a polyhedral cone which we rewrite as
              \[
                  \calK_j = \bigcap_{i=1}^n \cbr{w : [S_j]_{ii} \cdot \abr{x_i, w} \geq 0},
              \]
              where \( S_j = (2D_j - I) \).
    \end{itemize}

    \pause
    \horizontalrule

    \begin{beamercolorbox}[wd=\textwidth,sep=1em]{result}
        \textbf{Proposition 3.1} (informal): If \( X \) is full row-rank,
        then \( \text{aff}(\calK_j) = \R^d \) and
        \( \calK_j - \calK_j = \R^d \).
    \end{beamercolorbox}

    \vspace{1em}

    \pause
    Unfortunately, there is no extension to full-rank \( X \).

\end{frame}

\begin{frame}{Bonus: Not All Cones are Equal}

    \textbf{Alternative Idea}: show we don't need ``singular'' cones \( \calK_j \),
    \[
        \calK_j - \calK_j \subsetneq \R^d.
    \]

    \vspace{1em}
    \pause

    \begin{beamercolorbox}[wd=\textwidth,sep=1em]{result}
        \textbf{Proposition 3.2} (informal): Suppose \( \calK_j - \calK_j \subset \R^d \).
        Then, there exists \( \calK_i \) for which \( \calK_i - \calK_i = \R^d \)
        and \( \calK_j \subset \calK_i \).
    \end{beamercolorbox}

    \pause
    \vspace{1em}

    \textbf{Interpretation}: if optimal \( u^*_j \neq 0 \), then set
    \[
        u_i' = u_j^* + u_i^*.
    \]
    It is possible to show this causes no problems.

\end{frame}

\begin{frame}{Bonus: Cone Decomposition Proof Sketch}

    \textbf{Proof}: Works by iteratively constructing \( \calK_i \) s.t. \( \calK_j \subset \calK_i \).

    \pause
    \horizontalrule

    We sketch a simpler statement:

    \vspace{1em}

    \begin{beamercolorbox}[wd=\textwidth,sep=1em]{relaxation}
        \textbf{Proposition 3.2} (informal): Suppose \( \calK_j = \cbr{0} \).
        Then, there exists \( \calK_i \) for which \( \calK_i - \calK_i = \R^d \)
        and \( \calK_j \subset \calK_i \).
    \end{beamercolorbox}

\end{frame}

\begin{frame}{Bonus: Cone Decomposition Proof Sketch}
    \[
        \calK_j' = \cbr{w : [S_j]_{11} \cdot \abr{x_1, w} \geq 0}
    \]
    \begin{figure}[]
        \centering
        \input{assets/empty_cone_1}
    \end{figure}
\end{frame}

\begin{frame}{Cone Decompositions: Proof Sketch}

    \[
        \calK_j'' = \calK_j' \cap \cbr{w : [S_j]_{22} \cdot \abr{x_2, w} \geq 0}
    \]

    \begin{figure}[]
        \centering
        \input{assets/empty_cone_2}
    \end{figure}
\end{frame}

\begin{frame}{Bonus: Cone Decomposition Proof Sketch}

    \[
        \calK_j''' = \calK_j'' \cap \cbr{w : [S_j]_{33} \cdot \abr{x_3, w} \geq 0}
    \]

    \begin{figure}[]
        \centering
        \input{assets/empty_cone_3}
    \end{figure}
\end{frame}

\begin{frame}{Bonus: Cone Decomposition Proof Sketch}

    \[
        \tilde \calK_j''' = \calK_j'' \cap \cbr{w : -[S_j]_{33} \cdot \abr{x_3, w} \geq 0}
    \]

    \begin{figure}[]
        \centering
        %! TEX root = ../../main.tex

%% Illustration of cone decomposition. 

\begin{tikzpicture}[scale=1,
		declare function={
				cone_1(\x)= \x;
				cone_2(\x)= -\x;
				cone_3(\x)= -4*\x;
				bounds(\x)= -4*\x + 100;
			}
	]
	\begin{axis}[width=\linewidth, height=8cm,
			axis lines=center, yticklabels={,,}, xticklabels={,,},
			ymin=-4, ymax=4, ytick={-5,...,5}, ylabel=$$, x axis line style={-},
				xmin=-6, xmax=6, xtick={-5,...,5}, xlabel=$$, y axis line style={-},
		]
		\addplot[name path=cone_1, domain=-6:6, samples=100, line width=1pt]{cone_1(x)};
		\addplot[name path=cone_2, domain=-6:6, samples=200, line width=1pt]{cone_2(x)};
		\addplot[name path=cone_3, domain=-6:6, samples=200, line width=1pt, draw=purple]{cone_3(x)};
		\addplot[name path=bounds, domain=-6:6, samples=200, line width=1pt]{bounds(x)};

		% add color fill to both cones.
		\addplot fill between[
				of = cone_1 and cone_2,
				split, % calculate segments
				every even segment/.style = {fill=blue, fill opacity=0},
				every odd segment/.style  = {fill=teal, fill opacity=0.3}
			];

		\addplot fill between[
				of = cone_3 and bounds,
				every even segment/.style = {fill=orange, fill opacity=0.2},
			];

		%% point labels
		% origin point
		\node[circle, fill, inner sep=1pt] at (axis cs:0,0) {};

		% labels
		\node[label={0:$\calK_j$}] at (axis cs:4,-1) {};

		% lines
        \draw [->, draw=red, line width = 0.4mm] (axis cs:2,2) -- (axis cs:3,1) node[midway,above right] {$S_{11} \cdot x_1$};
		\draw [->, draw=red, line width = 0.4mm] (axis cs:2,-2) -- (axis cs:3,-1) node[midway,below right] {$S_{22} \cdot x_2$};
		\draw [->, draw=red, line width = 0.4mm] (axis cs:0.5,-2) -- (axis cs:1.5,-1.75) node[pos=0.2,below left] {$-S_{33} \cdot x_3$};
	\end{axis}

\end{tikzpicture}%

    \end{figure}
\end{frame}

\begin{frame}{Bonus: Main Cone Decomposition Result}
    \begin{itemize}
        \item The real proof is more complex, but this is the core idea.
              \vspace{0.2em}
              \begin{itemize}
                  \item Build \( \calK_i \) by switching signs of \( [S_j]_{ii} \).
                        \vspace{0.2em}
                  \item Equivalent to turning on/off activations.
              \end{itemize}
              \vspace{0.4em}

        \item Leads to our main approximation result.
    \end{itemize}

    \pause
    \horizontalrule

    \begin{beamercolorbox}[wd=\textwidth,sep=1em]{result}
        \textbf{Theorem 3.7} (informal):
        Let \( \lambda \geq 0 \) and let \( p^* \) be the optimal value of the ReLU problem.
        There exists a C-GReLU problem with minimizer \( u^* \) and optimal value \( d^* \) satisfying,
        \[
            d^* \leq p^* \leq d^* + \textcolor{Red}{2 \lambda \kappa(\tilde X_{\calJ}) \sum_{D_i \in \tilde \calD} \norm{u_i^*}_2}.
        \]
    \end{beamercolorbox}

\end{frame}

\setbeamercolor{background canvas}{bg=LightCyan}
\begin{frame}{}
    \begin{center}
        \huge Details on Optimization Algorithms
    \end{center}
\end{frame}
\setbeamercolor{background canvas}{bg=white}

\begin{frame}{Bonus: ReLU by Cone Decomposition}
    \begin{enumerate}
        \item Solve the gated ReLU problem:
              \[
                  u^* \in \argmin_{u} \norm{\sum_{j=1}^p D_j X u_j - y}_2^2 + \lambda \sum_{j=1}^p \norm{u_j}_2
              \]
              \pause
        \item Solve a cone decomposition:
              \[
                  v_j^*, w_j^* \in \argmin_{v_j, w_j} \cbr{ L(v_j, w_j) : v_j - w_j = u^*_j},
              \]
              where \( L \) is a loss function.
              \pause

        \item Compute corresponding ReLU model.
    \end{enumerate}

    \pause

    \vspace{2ex}
    \textbf{Choosing}:
    \begin{itemize}
        \item \( L(v, w) = \norm{v}_2 + \norm{w}_2 \) gives an SOCP.
              \pause

        \item \( L(v, w) = 0 \) yields a linear feasibility problem.
    \end{itemize}

\end{frame}

\begin{frame}{Bonus: R-FISTA}

    Consider ``composite'' optimization problem:
    \[
        \min_{x} f(x) + g(x),
    \]
    where \( f \) is \( L \)-smooth and \( g \) is convex.
    Smoothness implies
    \begin{equation*}
        \begin{aligned}
            f(y) & \leq Q_{\xk, 1/L}(y)                           \\
                 & = f(\xk)\! + \!\abr{\grad(\xk), y \!- \!\xk}\!
            +\! \frac{L}{2}\norm{y \!- \!\xk}_2^2.
        \end{aligned}
    \end{equation*}
    \vspace{2ex}

    \pause

    The \textbf{FISTA} algorithm minimizes \( Q_{\yk, \etak} \)
    and handles \( g \) exactly:
    \begin{align*}
        \xkk
         & = \argmin_{y} Q_{\yk, \etak}(y) + g(y)              \\
        \ykk
         & = \xkk + \frac{t_k - 1}{t_{k + 1}} \rbr{\xkk - \xk}
    \end{align*}
    where \( t_{k + 1} = (1 + \sqrt{1 + 4 t_k^2}) / 2 \).
\end{frame}

\begin{frame}{Bonus: R-FISTA Continued}
    We combine this with line-search and restarts:
    \vspace{1ex}

    \pause
    \begin{itemize}
        \item
              \textbf{Line-search}: backtrack on \( \etak \) until:
              \[
                  f(\xkk(\etak)) \leq Q_{\yk, \etak}(\xkk(\etak)),
              \]
              as proposed by \citep{beck2009fast}.
              \pause
              \vspace{1ex}
        \item
              \textbf{Restarts}: reset to \( \yk = \xk \) if
              \[ \abr{\xkk - \xk, \xkk - \yk} > 0, \]
              that is, \( \xkk \) is not a descent step
              with respect to proximal-gradient mapping
              ~\citep{odonoghue2015restarts}.

              \pause
              \vspace{1ex}
        \item And lots of other \textbf{convex tricks}...
    \end{itemize}
\end{frame}

\begin{frame}{Bonus: AL Method}
    Let \( \tilde X_i = (2 D_i - I) X \) so that
    \( v_j \in \calK_j \iff X_j v_j \geq 0 \) and define
    \[
        F(v, w) 
        = \norm{\sum_{j=1}^p D_j X (v_j - w_j) - y}_2^2 
        + \lambda \sum_{j=1}^p \norm{v_j}_2 + \norm{w_j}_2.
    \]
    Now we can form the augmented Lagrangian:
    \begin{equation}\label{eq:augmented-lagrangian}
        \begin{aligned}
            \!\!\!\!\calL_\delta & (v,\!w,\!\gamma,\!\zeta)\!:=\!(\delta / 2)\!\!\sum_{D_i \in \tilde \calD}\!\!\big[\|(\gamma_i / \delta\!-\! \tilde X_i v_i)_+\|_2^2 \\
                                 & \hspace{2em} + \|(\zeta_i / \delta - \tilde X_i w_i)_+\|_2^2 \big] + F(v,w).
        \end{aligned}
    \end{equation}
    We use the multiplier method to update the dual parameters:
    \begin{align*}
        \rbr{v_{k+1}, w_{k+1}}
                    & = \argmin_{v, w} \calL_{\delta}(v, w, \gamma_k, \zeta_k), \label{eq:al-subroutine} \\
        \gamma_{k + 1}
        = (\gamma_k & - \delta \tilde X_i v_i)_+, \quad
        \zeta_{k + 1} = (\zeta_k - \delta \tilde X_i w_i)_+. \nonumber
    \end{align*}

    \pause
    We use warm starts and propose a heuristic for \( \delta \).
\end{frame}

\setbeamercolor{background canvas}{bg=LightCyan}
\begin{frame}{}
    \begin{center}
        \huge Additional Optimization Experiments
    \end{center}
\end{frame}
\setbeamercolor{background canvas}{bg=white}

\begin{frame}{Bonus: Sub-sampling Patterns}

    \begin{figure}[t]
        \centering
        \includegraphics[width=0.6\textwidth]{assets/hp_paper.png}
    \end{figure}
    \begin{itemize}
        \item Variance induced by resampling \( \tilde \calD \) is minimal.
        \item Standard bias-variance trade-off.
    \end{itemize}
\end{frame}

\begin{frame}{Bonus: Generalization Performance}

    Generalization performance is equivalent to non-convex solvers.
    \begin{table}[t]
	\centering
	\begin{small}
		\begin{tabular}{lccc} \toprule
			\textbf{Dataset} & \textbf{Convex} & \textbf{Adam} & \textbf{SGD}  \\ \midrule
			magic            & 85.9            & \textbf{86.9} & 86.4          \\
			statlog-heart    & \textbf{83.3}   & \textbf{83.3} & 79.6          \\
            vertebral-col.   & \textbf{90.3}    & \textbf{90.3} & 88.7          \\
			cardiotocogr.    & \textbf{89.9}   & 36.5          & 88.9          \\
			abalone          & \textbf{66.2}   & 65.3          & 66.1          \\
			annealing        & 90.6            & \textbf{93.7} & 88.7          \\
			car              & 87.8            & \textbf{94.8} & 90.1          \\
			bank             & 89.8            & \textbf{90.8} & 90.5          \\
			breast-cancer    & \textbf{68.4}   & 64.9          & \textbf{68.4} \\
			page-blocks      & 94.0            & \textbf{97.1} & 96.9          \\
			contrac          & \textbf{55.1}   & 54.4          & 53.7          \\
			congressional    & 63.2            & 62.1          & \textbf{67.8} \\
			spambase         & 93.3            & \textbf{93.5} & 93.2          \\
			synthetic        & \textbf{98.3}   & 96.7          & 96.7          \\
			hill-valley      & \textbf{65.3}   & 62.8          & 55.4          \\ \bottomrule
		\end{tabular}
	\end{small}
\end{table}


\end{frame}

\begin{frame}{Bonus: Comparison to Standard Baselines}
    \setlength{\tabcolsep}{3pt} % Default value: 6pt
\begin{table}[t]
    \centering
    \begin{small}
        \begin{tabular}{lccccc} \toprule
            \small
            \textbf{Dataset} & \textbf{C-GReLU} & \textbf{C-ReLU} & \textbf{RF}   & \textbf{Linear} & \textbf{RBF}  \\ \midrule
            blood            & 79.9             & \textbf{80.5}   & 75.8          & 74.5            & 77.9          \\
            chess-krvkp      & \textbf{99.2}    & 98.6            & 98.9          & 97.2            & 98.4          \\
            conn-bench       & \textbf{90.2}    & 85.4            & 73.2          & 68.3            & 85.4          \\
            cylinder-bands   & 76.5             & \textbf{78.4}   & 77.5          & 71.6            & 71.6          \\
            fertility        & \textbf{80.0}    & \textbf{80.0}   & 75.0          & 75.0            & 75.0          \\
            heart-hung.      & \textbf{86.2}    & \textbf{86.2}   & 84.5          & 84.5            & \textbf{86.2} \\
            hill-valley      & \textbf{76.0}    & 68.6            & 57.9          & 62.0            & 70.2          \\
            ilpd-liver       & 72.4             & \textbf{74.1}   & 66.4          & 71.6            & 71.6          \\
            mammographic     & 77.6             & 78.6            & \textbf{80.7} & \textbf{80.7}   & 80.2          \\
            monks-1          & \textbf{100}     & \textbf{100}    & 95.8          & 79.2            & 83.3          \\
            musk-1           & 94.7             & \textbf{95.8}   & 92.6          & 86.3            & \textbf{95.8} \\
            ozone            & \textbf{97.6}    & \textbf{97.6}   & 97.4          & 97.2            & 97.4          \\
            pima             & 74.5             & 74.5            & \textbf{76.5} & 75.2            & 73.2          \\
            planning         & \textbf{69.4}    & 63.9            & 66.7          & 66.7            & \textbf{69.4} \\
            spambase         & 93.5             & 93.6            & \textbf{94.1} & 92.2            & 93.6          \\
            spectf           & \textbf{87.5}    & 75.0            & 68.8          & 68.8            & 68.8          \\
            statlog-german   & 74.0             & \textbf{77.5}   & 73.5          & 75.0            & 75.5          \\
            tic-tac-toe      & 99.0             & 99.0            & 99.5          & 98.4            & \textbf{100}  \\ \bottomrule
        \end{tabular}
    \end{small}
\end{table}
\setlength{\tabcolsep}{6pt}

% convex is best or ties on 24 & 38 datasets. Next best is the kernel SVM with 12 & 38 datasets.


\end{frame}

\begin{frame}{Bonus: Acceleration Ablation}
    \begin{figure}
        \centering
        \includegraphics[width=0.75\textwidth]{assets/pp_acceleration.pdf}
    \end{figure}
\end{frame}

